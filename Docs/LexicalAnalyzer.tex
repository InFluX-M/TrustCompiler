\documentclass[12pt, a4paper]{report}
\usepackage{hyperref}
\usepackage[top=3cm,right=3cm,bottom=3cm,left=2.5cm]{geometry}
\usepackage{caption}
\usepackage{indentfirst}
\usepackage{graphicx}
\usepackage{subfigure}
\usepackage{float}
\usepackage{fancyhdr}
\usepackage{titlesec}
\usepackage{xepersian}
\settextfont{XB Zar}
\linespread{1.2}
\setcounter{secnumdepth}{4}
\setcounter{tocdepth}{4}


\pagestyle{fancy}
% Set up fancyhdr to place page number at the top left
\pagestyle{fancy}
\fancyhf{} % Clear all header/footer fields
\fancyhead[L]{\thepage} % Page number on the top left

% Define custom style for chapter page (no header or footer)
\fancypagestyle{plain}{
	\fancyhf{} % Clear headers and footers
	\renewcommand{\headrulewidth}{0pt} % Remove header rule
}


% Changing TOC addressing format (from period separation to dash seperation)
\renewcommand{\thesection}{\thechapter-\arabic{section}-}
\renewcommand{\thesubsection}{\thechapter-\arabic{section}-\arabic{subsection}-}
\renewcommand{\thesubsubsection}{\thechapter-\arabic{section}-\arabic{subsection}-\arabic{subsubsection}-}

\renewcommand{\thetable}{(\thechapter-\arabic{table})}

% Redefine how captions are written 
\renewcommand{\thefigure}{(\thechapter-\arabic{figure})}

% Set caption font size to 11pt
\captionsetup{font=small, labelsep=space} % 'small' is equivalent to 11pt in most classes

% Make sure Persian and Latin text have their respective font sizes
\settextfont{XB Zar} % Persian font
\setlatintextfont{Times New Roman} % Latin font

% Automatically switch between Persian and Latin fonts based on character type
\XeTeXinterchartokenstate=1
\newXeTeXintercharclass\persianchars
\newXeTeXintercharclass\latinchars

% Automatically switch fonts when switching between Persian and Latin characters
\XeTeXinterchartoks \latinchars \persianchars = {\begingroup\persianfont\endgroup}
\XeTeXinterchartoks \persianchars \latinchars = {\begingroup\latinfont\endgroup}



\title{گزارش کارآموزی}
\author{متین اعظمی}

\begin{document}
%\maketitle
\begin{titlepage}
	\centering
	% University logo
	\includegraphics[width=0.15\textwidth]{images/logo.png}
	
	% University and Faculty name
	{\Large دانشگاه اصفهان}\par
	{\Large دانشکده مهندسی کامپیوتر}\par\vspace{2cm}
	
	% Report title
	\textbf
	{\Huge گزارش پروژه کامپایلر}\par\vspace{1.5cm}
	
	% Intern details
	\large
	\textbf{نام و نام خانوادگی اعضا:}\par{زهرا معصومی / متین اعظمی}\par\vspace{0.5cm}  
	\textbf{شماره دانشجویی:}\par{۴۰۰۳۶۲۳۰۰۳ / ۳۰۰۳۶۲۳۱۳۳۴}\par\vspace{0.5cm}
	\textbf{استاد کارآموزی:}\par{دکتر شفیعی}\par\vspace{0.5cm}
		
	\par\vspace{2cm}
	
	
\end{titlepage}

\tableofcontents

\listoftables

\listoffigures


\chapter{تحلیل‌گر لغوی}
\section{توابع تحلیلگر لغوی}

\subsection{تابع \lr{is\_keyword}}
\label{subsec:is_keyword}

\subsubsection*{هدف:}
این تابع بررسی می‌کند که آیا دنباله‌ای از کاراکترها در خط جاری با یکی از کلمات کلیدی زبان Trust مطابقت دارد یا خیر.

\subsubsection*{ورودی‌ها:}
\begin{itemize}
\item \lr{index:} اشاره‌گر به موقعیت فعلی در خط
\item \lr{line:} رشته حاوی کد منبع
\item \lr{line\_number:} شماره خط فعلی برای گزارش خطاها
\end{itemize}

\subsubsection*{خروجی:}
\begin{itemize}
\item توکن مربوط به کلمه کلیدی در صورت تطابق
\item توکن \lr{Invalid} در صورت عدم تطابق
\end{itemize}

\subsubsection*{الگوریتم:}
\begin{enumerate}
\item بررسی لیست کلمات کلیدی از پیش تعریف شده
\item مقایسه زیررشته شروع شده از \lr{index} با هر کلمه کلیدی
\item بررسی عدم وجود حروف/ارقام بعد از کلمه کلیدی (برای جلوگیری از شناسایی نادرست شناسه)
\item بازگرداندن توکن مربوطه در صورت تطابق کامل
\end{enumerate}

\subsubsection*{نکات مهم:}
\begin{itemize}
\item حساس به حروف کوچک و بزرگ
\item کلمات کلیدی باید به صورت کامل و مجزا ظاهر شوند
\item لیست کلمات کلیدی شامل: \lr{bool, break, continue, else, false, fn, i32, if, let, loop, mut, println!, return, true}
\end{itemize}

\subsection{تابع \lr{is\_id}}
\label{subsec:is_id}

\subsubsection*{هدف:}
تشخیص شناسه‌های معتبر طبق قوانین زبان Trust

\subsubsection*{ورودی‌ها:} مشابه بخش \ref{subsec:is_keyword}

\subsubsection*{خروجی:}
\begin{itemize}
\item توکن \lr{T\_Id} برای شناسه‌های معتبر
\item توکن \lr{Invalid} برای موارد نامعتبر
\end{itemize}

\subsubsection*{الگوریتم:}
\begin{enumerate}
\item بررسی شروع شناسه با حرف یا زیرخط (\_)
\item بررسی ادامه شناسه با حروف، اعداد یا زیرخط
\item مقایسه با کلمات کلیدی برای جلوگیری از تداخل
\item بازگرداندن توکن مناسب بر اساس نتایج
\end{enumerate}

\subsubsection*{نکات مهم:}
\begin{itemize}
\item شناسه‌ها نمی‌توانند با اعداد شروع شوند
\item طول شناسه محدودیتی ندارد
\item حساس به حروف کوچک و بزرگ
\end{itemize}

\subsection{تابع \lr{is\_decimal}}
\label{subsec:is_decimal}

\subsubsection*{هدف:}
تشخیص اعداد صحیح دهدهی (مثبت/منفی)

\subsubsection*{الگوریتم:}
\begin{enumerate}
\item بررسی وجود علامت منفی اختیاری
\item بررسی دنباله‌ای از ارقام 0-9
\item بازگرداندن توکن \lr{T\_Decimal} برای اعداد معتبر
\end{enumerate}

\subsubsection*{نکات:}
\begin{itemize}
\item حداقل یک رقم الزامی است
\item اجازه استفاده از صفر پیشرو وجود دارد
\item محدوده عددی بررسی نمی‌شود
\end{itemize}

\subsection{تابع \lr{is\_hexadecimal}}
\label{subsec:is_hexadecimal}

\subsubsection*{هدف:}
تشخیص اعداد هگزادسیمال با فرمت \lr{0x...}

\subsubsection*{الگوریتم:}
\begin{enumerate}
\item بررسی پیشوند \lr{0x} یا \lr{0X}
\item بررسی حداقل یک رقم هگزادسیمال (0-9, a-f, A-F)
\item بازگرداندن توکن \lr{T\_Hexadecimal} برای اعداد معتبر
\end{enumerate}

\subsubsection*{نکات:}
\begin{itemize}
\item حروف بزرگ و کوچک در ارقام تفاوتی ندارند
\item عدم پشتیبانی از اعداد منفی
\item کاراکترهای غیر هگزادسیمال باعث پایان عدد می‌شوند
\end{itemize}

\section{نتیجه‌گیری}
این توابع پایه‌ای اساسی برای تجزیه lexical کد منبع فراهم می‌کنند. هر تابع با استفاده از ماشین حالت محدود (Finite State Machine) پیاده‌سازی شده و قادر به تشخیص الگوهای پیچیده با کارایی بالا می‌باشد. تعامل صحیح بین این توابع تضمین می‌کند که تمامی اجزای برنامه ورودی به درستی شناسایی و طبقه‌بندی می‌شوند.



\end{document}